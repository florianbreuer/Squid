\documentclass{article}
\usepackage{amssymb,amsmath,hyperref,a4wide,longtable,graphicx}

\begin{document}\renewcommand{\arraystretch}{1.5}\setcounter{page}{0}
{\Large MATH1120-LS-2022 W2}

Marking scheme for written-answer question

\setcounter{section}{-1}

\section{Variant List}

\medskip
\begin{longtable}{|l|l|l|}
\hline
Variant & $n$th term & Radius\\ \hline
\hyperref[v1]{Variant 1} & $n\left(1/2\right)^n \big(3x + 2\big)^n$ & $2/3$\\ \hline
\hyperref[v2]{Variant 2} & $n\left(1/2\right)^n \big(4x + 2\big)^n$ & $1/2$\\ \hline
\hyperref[v3]{Variant 3} & $n\left(1/2\right)^n \big(5x + 2\big)^n$ & $2/5$\\ \hline
\hyperref[v4]{Variant 4} & $n\left(1/3\right)^n \big(2x + 2\big)^n$ & $3/2$\\ \hline
\hyperref[v5]{Variant 5} & $n\left(1/3\right)^n \big(4x + 2\big)^n$ & $3/4$\\ \hline
\hyperref[v6]{Variant 6} & $n\left(1/3\right)^n \big(5x + 2\big)^n$ & $3/5$\\ \hline
\hyperref[v7]{Variant 7} & $n\left(1/4\right)^n \big(3x + 2\big)^n$ & $4/3$\\ \hline
\hyperref[v8]{Variant 8} & $n\left(1/4\right)^n \big(5x + 2\big)^n$ & $4/5$\\ \hline
\hyperref[v9]{Variant 9} & $n\left(1/5\right)^n \big(2x + 2\big)^n$ & $5/2$\\ \hline
\hyperref[v10]{Variant 10} & $n\left(1/5\right)^n \big(3x + 2\big)^n$ & $5/3$\\ \hline
\hyperref[v11]{Variant 11} & $n\left(1/5\right)^n \big(4x + 2\big)^n$ & $5/4$\\ \hline
\end{longtable}

\medskip
\noindent{\bf Marking Scheme:}
            \begin{small}
            \begin{itemize}
            \item 1 mark: The student demonstrates a partial understanding of how to do the problem.
            \item 2 marks: The student demonstrates a good understanding of how to do the problem \\ (some minor errors permitted).
            \item 3 marks: The student demonstrates a good understanding and obtains the correct answer.
            \end{itemize}
            \end{small}


        \newpage
        \section{Variant 1}
        \label{v1}


Compute the radius of convergence of the following power series. Show all your work.
    \[
    f(x) = \sum_{n=0}^\infty n\left(1/2\right)^n \big(3x + 2\big)^n
    \]


[For office use only: V1]
        \medskip

        \noindent{\bf Solution.}


    We use the ratio test. Let
      \begin{align*}
        L &= \lim_{n\to\infty} \left| 
        \frac{(n+1)\left(1/2\right)^{n+1}\big(3x + 2\big)^{n+1}}
        {n\left(1/2\right)^n \big(3x + 2\big)^n}
        \right|\\
        &= \lim_{n\to\infty} \left| \frac{n+1}{n}(1/2)(3x+2)\right| \\
        &= \left| (1/2)(3x+2)\right|.
       \end{align*}
       The power series will converge if $L < 1$, i.e.
       \[
        \left| (1/2)(3x+2)\right| < 1 \qquad\Longleftrightarrow \qquad
        \left|x + 2/3\right| < 2/3.
       \]
       Thus, the radius of convergence is $R = 2/3$.
        \medskip

        \noindent{\bf Marking Scheme:}
            \begin{small}
            \begin{itemize}
            \item 1 mark: The student demonstrates a partial understanding of how to do the problem.
            \item 2 marks: The student demonstrates a good understanding of how to do the problem \\ (some minor errors permitted).
            \item 3 marks: The student demonstrates a good understanding and obtains the correct answer.
            \end{itemize}
            \end{small}


        \newpage
        \section{Variant 2}
        \label{v2}


Compute the radius of convergence of the following power series. Show all your work.
    \[
    f(x) = \sum_{n=0}^\infty n\left(1/2\right)^n \big(4x + 2\big)^n
    \]


[For office use only: V2]
        \medskip

        \noindent{\bf Solution.}


    We use the ratio test. Let
      \begin{align*}
        L &= \lim_{n\to\infty} \left| 
        \frac{(n+1)\left(1/2\right)^{n+1}\big(4x + 2\big)^{n+1}}
        {n\left(1/2\right)^n \big(4x + 2\big)^n}
        \right|\\
        &= \lim_{n\to\infty} \left| \frac{n+1}{n}(1/2)(4x+2)\right| \\
        &= \left| (1/2)(4x+2)\right|.
       \end{align*}
       The power series will converge if $L < 1$, i.e.
       \[
        \left| (1/2)(4x+2)\right| < 1 \qquad\Longleftrightarrow \qquad
        \left|x + 1/2\right| < 1/2.
       \]
       Thus, the radius of convergence is $R = 1/2$.
        \medskip

        \noindent{\bf Marking Scheme:}
            \begin{small}
            \begin{itemize}
            \item 1 mark: The student demonstrates a partial understanding of how to do the problem.
            \item 2 marks: The student demonstrates a good understanding of how to do the problem \\ (some minor errors permitted).
            \item 3 marks: The student demonstrates a good understanding and obtains the correct answer.
            \end{itemize}
            \end{small}


        \newpage
        \section{Variant 3}
        \label{v3}


Compute the radius of convergence of the following power series. Show all your work.
    \[
    f(x) = \sum_{n=0}^\infty n\left(1/2\right)^n \big(5x + 2\big)^n
    \]


[For office use only: V3]
        \medskip

        \noindent{\bf Solution.}


    We use the ratio test. Let
      \begin{align*}
        L &= \lim_{n\to\infty} \left| 
        \frac{(n+1)\left(1/2\right)^{n+1}\big(5x + 2\big)^{n+1}}
        {n\left(1/2\right)^n \big(5x + 2\big)^n}
        \right|\\
        &= \lim_{n\to\infty} \left| \frac{n+1}{n}(1/2)(5x+2)\right| \\
        &= \left| (1/2)(5x+2)\right|.
       \end{align*}
       The power series will converge if $L < 1$, i.e.
       \[
        \left| (1/2)(5x+2)\right| < 1 \qquad\Longleftrightarrow \qquad
        \left|x + 2/5\right| < 2/5.
       \]
       Thus, the radius of convergence is $R = 2/5$.
        \medskip

        \noindent{\bf Marking Scheme:}
            \begin{small}
            \begin{itemize}
            \item 1 mark: The student demonstrates a partial understanding of how to do the problem.
            \item 2 marks: The student demonstrates a good understanding of how to do the problem \\ (some minor errors permitted).
            \item 3 marks: The student demonstrates a good understanding and obtains the correct answer.
            \end{itemize}
            \end{small}


        \newpage
        \section{Variant 4}
        \label{v4}


Compute the radius of convergence of the following power series. Show all your work.
    \[
    f(x) = \sum_{n=0}^\infty n\left(1/3\right)^n \big(2x + 2\big)^n
    \]


[For office use only: V4]
        \medskip

        \noindent{\bf Solution.}


    We use the ratio test. Let
      \begin{align*}
        L &= \lim_{n\to\infty} \left| 
        \frac{(n+1)\left(1/3\right)^{n+1}\big(2x + 2\big)^{n+1}}
        {n\left(1/3\right)^n \big(2x + 2\big)^n}
        \right|\\
        &= \lim_{n\to\infty} \left| \frac{n+1}{n}(1/3)(2x+2)\right| \\
        &= \left| (1/3)(2x+2)\right|.
       \end{align*}
       The power series will converge if $L < 1$, i.e.
       \[
        \left| (1/3)(2x+2)\right| < 1 \qquad\Longleftrightarrow \qquad
        \left|x + 1\right| < 3/2.
       \]
       Thus, the radius of convergence is $R = 3/2$.
        \medskip

        \noindent{\bf Marking Scheme:}
            \begin{small}
            \begin{itemize}
            \item 1 mark: The student demonstrates a partial understanding of how to do the problem.
            \item 2 marks: The student demonstrates a good understanding of how to do the problem \\ (some minor errors permitted).
            \item 3 marks: The student demonstrates a good understanding and obtains the correct answer.
            \end{itemize}
            \end{small}


        \newpage
        \section{Variant 5}
        \label{v5}


Compute the radius of convergence of the following power series. Show all your work.
    \[
    f(x) = \sum_{n=0}^\infty n\left(1/3\right)^n \big(4x + 2\big)^n
    \]


[For office use only: V5]
        \medskip

        \noindent{\bf Solution.}


    We use the ratio test. Let
      \begin{align*}
        L &= \lim_{n\to\infty} \left| 
        \frac{(n+1)\left(1/3\right)^{n+1}\big(4x + 2\big)^{n+1}}
        {n\left(1/3\right)^n \big(4x + 2\big)^n}
        \right|\\
        &= \lim_{n\to\infty} \left| \frac{n+1}{n}(1/3)(4x+2)\right| \\
        &= \left| (1/3)(4x+2)\right|.
       \end{align*}
       The power series will converge if $L < 1$, i.e.
       \[
        \left| (1/3)(4x+2)\right| < 1 \qquad\Longleftrightarrow \qquad
        \left|x + 1/2\right| < 3/4.
       \]
       Thus, the radius of convergence is $R = 3/4$.
        \medskip

        \noindent{\bf Marking Scheme:}
            \begin{small}
            \begin{itemize}
            \item 1 mark: The student demonstrates a partial understanding of how to do the problem.
            \item 2 marks: The student demonstrates a good understanding of how to do the problem \\ (some minor errors permitted).
            \item 3 marks: The student demonstrates a good understanding and obtains the correct answer.
            \end{itemize}
            \end{small}


        \newpage
        \section{Variant 6}
        \label{v6}


Compute the radius of convergence of the following power series. Show all your work.
    \[
    f(x) = \sum_{n=0}^\infty n\left(1/3\right)^n \big(5x + 2\big)^n
    \]


[For office use only: V6]
        \medskip

        \noindent{\bf Solution.}


    We use the ratio test. Let
      \begin{align*}
        L &= \lim_{n\to\infty} \left| 
        \frac{(n+1)\left(1/3\right)^{n+1}\big(5x + 2\big)^{n+1}}
        {n\left(1/3\right)^n \big(5x + 2\big)^n}
        \right|\\
        &= \lim_{n\to\infty} \left| \frac{n+1}{n}(1/3)(5x+2)\right| \\
        &= \left| (1/3)(5x+2)\right|.
       \end{align*}
       The power series will converge if $L < 1$, i.e.
       \[
        \left| (1/3)(5x+2)\right| < 1 \qquad\Longleftrightarrow \qquad
        \left|x + 2/5\right| < 3/5.
       \]
       Thus, the radius of convergence is $R = 3/5$.
        \medskip

        \noindent{\bf Marking Scheme:}
            \begin{small}
            \begin{itemize}
            \item 1 mark: The student demonstrates a partial understanding of how to do the problem.
            \item 2 marks: The student demonstrates a good understanding of how to do the problem \\ (some minor errors permitted).
            \item 3 marks: The student demonstrates a good understanding and obtains the correct answer.
            \end{itemize}
            \end{small}


        \newpage
        \section{Variant 7}
        \label{v7}


Compute the radius of convergence of the following power series. Show all your work.
    \[
    f(x) = \sum_{n=0}^\infty n\left(1/4\right)^n \big(3x + 2\big)^n
    \]


[For office use only: V7]
        \medskip

        \noindent{\bf Solution.}


    We use the ratio test. Let
      \begin{align*}
        L &= \lim_{n\to\infty} \left| 
        \frac{(n+1)\left(1/4\right)^{n+1}\big(3x + 2\big)^{n+1}}
        {n\left(1/4\right)^n \big(3x + 2\big)^n}
        \right|\\
        &= \lim_{n\to\infty} \left| \frac{n+1}{n}(1/4)(3x+2)\right| \\
        &= \left| (1/4)(3x+2)\right|.
       \end{align*}
       The power series will converge if $L < 1$, i.e.
       \[
        \left| (1/4)(3x+2)\right| < 1 \qquad\Longleftrightarrow \qquad
        \left|x + 2/3\right| < 4/3.
       \]
       Thus, the radius of convergence is $R = 4/3$.
        \medskip

        \noindent{\bf Marking Scheme:}
            \begin{small}
            \begin{itemize}
            \item 1 mark: The student demonstrates a partial understanding of how to do the problem.
            \item 2 marks: The student demonstrates a good understanding of how to do the problem \\ (some minor errors permitted).
            \item 3 marks: The student demonstrates a good understanding and obtains the correct answer.
            \end{itemize}
            \end{small}


        \newpage
        \section{Variant 8}
        \label{v8}


Compute the radius of convergence of the following power series. Show all your work.
    \[
    f(x) = \sum_{n=0}^\infty n\left(1/4\right)^n \big(5x + 2\big)^n
    \]


[For office use only: V8]
        \medskip

        \noindent{\bf Solution.}


    We use the ratio test. Let
      \begin{align*}
        L &= \lim_{n\to\infty} \left| 
        \frac{(n+1)\left(1/4\right)^{n+1}\big(5x + 2\big)^{n+1}}
        {n\left(1/4\right)^n \big(5x + 2\big)^n}
        \right|\\
        &= \lim_{n\to\infty} \left| \frac{n+1}{n}(1/4)(5x+2)\right| \\
        &= \left| (1/4)(5x+2)\right|.
       \end{align*}
       The power series will converge if $L < 1$, i.e.
       \[
        \left| (1/4)(5x+2)\right| < 1 \qquad\Longleftrightarrow \qquad
        \left|x + 2/5\right| < 4/5.
       \]
       Thus, the radius of convergence is $R = 4/5$.
        \medskip

        \noindent{\bf Marking Scheme:}
            \begin{small}
            \begin{itemize}
            \item 1 mark: The student demonstrates a partial understanding of how to do the problem.
            \item 2 marks: The student demonstrates a good understanding of how to do the problem \\ (some minor errors permitted).
            \item 3 marks: The student demonstrates a good understanding and obtains the correct answer.
            \end{itemize}
            \end{small}


        \newpage
        \section{Variant 9}
        \label{v9}


Compute the radius of convergence of the following power series. Show all your work.
    \[
    f(x) = \sum_{n=0}^\infty n\left(1/5\right)^n \big(2x + 2\big)^n
    \]


[For office use only: V9]
        \medskip

        \noindent{\bf Solution.}


    We use the ratio test. Let
      \begin{align*}
        L &= \lim_{n\to\infty} \left| 
        \frac{(n+1)\left(1/5\right)^{n+1}\big(2x + 2\big)^{n+1}}
        {n\left(1/5\right)^n \big(2x + 2\big)^n}
        \right|\\
        &= \lim_{n\to\infty} \left| \frac{n+1}{n}(1/5)(2x+2)\right| \\
        &= \left| (1/5)(2x+2)\right|.
       \end{align*}
       The power series will converge if $L < 1$, i.e.
       \[
        \left| (1/5)(2x+2)\right| < 1 \qquad\Longleftrightarrow \qquad
        \left|x + 1\right| < 5/2.
       \]
       Thus, the radius of convergence is $R = 5/2$.
        \medskip

        \noindent{\bf Marking Scheme:}
            \begin{small}
            \begin{itemize}
            \item 1 mark: The student demonstrates a partial understanding of how to do the problem.
            \item 2 marks: The student demonstrates a good understanding of how to do the problem \\ (some minor errors permitted).
            \item 3 marks: The student demonstrates a good understanding and obtains the correct answer.
            \end{itemize}
            \end{small}


        \newpage
        \section{Variant 10}
        \label{v10}


Compute the radius of convergence of the following power series. Show all your work.
    \[
    f(x) = \sum_{n=0}^\infty n\left(1/5\right)^n \big(3x + 2\big)^n
    \]


[For office use only: V10]
        \medskip

        \noindent{\bf Solution.}


    We use the ratio test. Let
      \begin{align*}
        L &= \lim_{n\to\infty} \left| 
        \frac{(n+1)\left(1/5\right)^{n+1}\big(3x + 2\big)^{n+1}}
        {n\left(1/5\right)^n \big(3x + 2\big)^n}
        \right|\\
        &= \lim_{n\to\infty} \left| \frac{n+1}{n}(1/5)(3x+2)\right| \\
        &= \left| (1/5)(3x+2)\right|.
       \end{align*}
       The power series will converge if $L < 1$, i.e.
       \[
        \left| (1/5)(3x+2)\right| < 1 \qquad\Longleftrightarrow \qquad
        \left|x + 2/3\right| < 5/3.
       \]
       Thus, the radius of convergence is $R = 5/3$.
        \medskip

        \noindent{\bf Marking Scheme:}
            \begin{small}
            \begin{itemize}
            \item 1 mark: The student demonstrates a partial understanding of how to do the problem.
            \item 2 marks: The student demonstrates a good understanding of how to do the problem \\ (some minor errors permitted).
            \item 3 marks: The student demonstrates a good understanding and obtains the correct answer.
            \end{itemize}
            \end{small}


        \newpage
        \section{Variant 11}
        \label{v11}


Compute the radius of convergence of the following power series. Show all your work.
    \[
    f(x) = \sum_{n=0}^\infty n\left(1/5\right)^n \big(4x + 2\big)^n
    \]


[For office use only: V11]
        \medskip

        \noindent{\bf Solution.}


    We use the ratio test. Let
      \begin{align*}
        L &= \lim_{n\to\infty} \left| 
        \frac{(n+1)\left(1/5\right)^{n+1}\big(4x + 2\big)^{n+1}}
        {n\left(1/5\right)^n \big(4x + 2\big)^n}
        \right|\\
        &= \lim_{n\to\infty} \left| \frac{n+1}{n}(1/5)(4x+2)\right| \\
        &= \left| (1/5)(4x+2)\right|.
       \end{align*}
       The power series will converge if $L < 1$, i.e.
       \[
        \left| (1/5)(4x+2)\right| < 1 \qquad\Longleftrightarrow \qquad
        \left|x + 1/2\right| < 5/4.
       \]
       Thus, the radius of convergence is $R = 5/4$.
        \medskip

        \noindent{\bf Marking Scheme:}
            \begin{small}
            \begin{itemize}
            \item 1 mark: The student demonstrates a partial understanding of how to do the problem.
            \item 2 marks: The student demonstrates a good understanding of how to do the problem \\ (some minor errors permitted).
            \item 3 marks: The student demonstrates a good understanding and obtains the correct answer.
            \end{itemize}
            \end{small}

\end{document}